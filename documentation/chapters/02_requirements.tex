\chapter{Requirement Analysis and Specification}

The MLUD will offer the following functionalities:
\begin{itemize}
    \item User registration and book detail submission via an online form.
    \item Facilitating the physical handover of books, including verification and labeling.
    \item Streamlining the sales process with a digital interface for managing transactions.
    \item Tracking unsold books and managing financial settlements for sellers.
    \item Generating aggregated statistical reports to monitor performance and user satisfaction.
\end{itemize}

\section{Hardware Interfaces}
The system will be accessible through a web interface. The PR will have the option to use both a computer and a mobile device to access the system, so the interface must be responsive and usable on both platforms.

The OP will use a computer to access the system, so the interface must be optimized for computer use.

\section{Use Cases}

The use cases are illustrated in the diagram in Figure \ref{fig:use_cases}.

\begin{figure}[h]
    \centering
    \includegraphics[width=.75\textwidth]{assets/use_cases_diagram.png}
    \caption{Use cases diagram}
    \label{fig:use_cases}
\end{figure}

\subsection{Use Case 1: Book submission}

\begin{enumerate}
    \item The PR accesses the system.
    \item The PR fills out the form, providing:
          \begin{itemize}
              \item Personal information (name, surname, email, phone number, school of origin).
              \item Information for an arbitrary number of books (ISBN, title, author, edition, price, condition).
              \item Acceptance of the terms and conditions.
              \item Acceptance of Lokalino rules.
              \item Preference for subscription to the mailing list.
              \item Approximate period of delivery of the books.
          \end{itemize}
          When a period of delivery is selected, the system will show the number od PR that have already submitted the form for that period, so that the PR can choose a period with a reasonable number of PRs.\\
          When entering the ISBN or the title of the book, the system will suggest books that are already in the system (that are books that have been adopted by the schools in the area), so that the PR can select them and avoid entering the same book multiple times. If the book is not in the system, the PR cannot enter it manually, because most likely it is not adopted anymore, but the PR can still bring it to Lokalino and the OP will be able to verify it and add it to the system if it is correct.
    \item The PR submits the form, it is redirected to a confirmation page.
\end{enumerate}

\subsection{Use Case 2: Book delivery}

\begin{enumerate}
    \item The PR goes to Lokalino with the books.
    \item The OP accesses the protected area of the system.
    \item The OP searches for the record of the PR in the system.
    \item The OP verifies the correspondence between the books and the records, eventually updating the records or adding comments
    \item If all is correct, the OP labels the books to identify the corresponding PR
    \item The OP mark the correct delivery of the books in the system
\end{enumerate}

\subsection{Use Case 3: Book sale}

\begin{enumerate}
    \item The OP accesses the protected area of the system.
    \item The BY selects the books they want to buy.
    \item For each book, the OP searches for the record in the system and adds the book to the cart.
    \item At the end of the selection, the OP confirms the purchase and collects the payment. The books are marked as sold in the system.
\end{enumerate}

\subsection{Use Case 4: Liquidation and return of unsold books}

\begin{enumerate}
    \item The OP accesses the protected area of the system.
    \item The OP searches for the record of the PR in the system.
    \item The system shows the list of books unsold and the total amount due to the PR.
    \item The PR collects the money and the unsold books.
    \item The OP marks the books as returned in the system.
\end{enumerate}

\section{Special Requirements}

\begin{itemize}
    \item In order to generate the list of books that can be suggested to the PR, the system will provide the possibility to the OP to insert the list of books that have been adopted by the schools in the area. The OP will be able to insert the list of books in a facilitated way, to be defined directly with the OP a short time before the delivery.
    \item The system will keep the data of the PRs and the books for 1 year after the event, and after that, it will be deleted. The only data that will be kept indefinitely is the list of persons who have subscribed to the mailing list.
    \item Every time an OP handles information about a PR, the system will also show the unique ID of the PR, so that the OP can easily label and identify the books.
    \item Since the prices of the books are different from year to year, the system will not compile automatically the price of the books when the PR fills out the form, as it does for the Author and Editor. The PR will have to fill out the price of the book manually, and the OP will have to verify it when the books are delivered. If the price is not correct, the OP will have to delete the book from the system and re-enter it with the correct price.
\end{itemize}

\section{Assumptions}
\label{sec:assumptions}

An assumptions have been made in the design of the system:

Even if in the implementation every PR will be identified by a unique ID, the system will check the uniqueness of the email address; this comes into play when a PR submits the form: if the email address is already in the system and the PR fills out the form again, the system will not create a new record and will work as if the same PR is inserting more books.